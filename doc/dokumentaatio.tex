\documentclass[a4paper,titlepage]{article}
\usepackage{fancyhdr}
\usepackage{hyperref}
\usepackage[utf8]{inputenc}
\usepackage[T1]{fontenc}
\usepackage[finnish]{babel}

\hyphenation{kahvi-päivä-kirjaan}

\title{Kahvipäiväkirja}
\author{Miikka Koskinen \\ \url{miikka.koskinen@iki.fi}}

\begin{document}
\maketitle
\tableofcontents
\newpage

\section{Johdanto}

Juon mielelläni hyvää kahvia. En kuitenkaan oikeastaan osaa sanoa, että mikä
tekee kahvista hyvää tai millaisesta kahvista pidän. Siksi aion alkaa pitämään
kahvipäiväkirjaa, johon merkitsen, mitä kahvia olen juonut ja miltä se maistui.
Tämän työn tavoitteena on toteuttaa web-pohjainen kahvipäiväkirja.

Sovelluksen tarkoituksena on antaa käyttäjän kirjata
maistelukokemuksia. Olennaista tietoa on, mitä kahvia
maisteltiin. Käyttäjä voi antaa kahville arvosanan ja kirjata
vapaamuotoisia huomioita. Lisäksi sovellus tarjoaa listan parhaista
kahveista ja kahvityypeistä.

\subsection{Päätoiminnallisuudet}

\begin{itemize}
\item Maistelukokemuksien kirjaaminen ja muokkaaminen.
\item Maisteluhistorian ja top-listojen katselu.
\item Käyttäjän ja ylläpitäjän kirjautuminen.
\item Syötetyn datan siivoaminen ylläpitäjän toimesta.
\end{itemize}

\subsection{Toteutustekniikat}

Ohjelmointikielenä on Clojure. Clojure on moderni, dynaaminen,
Lisp-tyylinen kieli JVM-alustalle. Sovellusta ajetaan laitoksen
users-palvelimen Tomcatin alla WAR-paketiksi käännettynä. Tietokantana
toimii PostgreSQL.

Työssä ei käytetä yhtä isoa web-sovelluskehystä, vaan kokoelmaa pieniä
Clojure-kirjastoja, kuten Compojure-reitityskirjastoa ja
Hiccup-mallinekirjastoa. Yhdessä kirjastot tarjoavat samankaltaisen
ympäristön kuin Sinatra Ruby-maailmassa. Tietokantaan yhdistetään
JDBC:n avulla, joten myös muun kuin PostgreSQL-tietokannan käyttö on
mahdollista.

Sovelluksen on tarkoitus toimia työpöytäkoneiden lisäksi iPhonen
Safari-selaimella, jotta maistelukokemuksia on helppo kirjata
esimerkiksi kahvilasta käsin.

\section{Yleiskuvaus}

Käyttäjäryhmät:

\begin{description}
    \item[Jokamies]
        Jokamies on kuka tahansa, joka vierailee kahvipäiväkirjassa. Kaikki
        muut käyttäjäryhmät kuuluvat myös tähän käyttäjäryhmään.

    \item[Maistelija]
        Maistelija on rekisteröity käyttäjä. Maistelija voi lisätä
        kahvipäiväkirjaan omia maistelukokemuksiaan.

    \item[Ylläpitäjä]
        Ylläpitäjä on rekisteröity käyttäjä, joka hallinnoi käyttäjätilejä ja
        joka pitää huolta siitä, että järjestelmään syötetty tieto on
        laadukasta.
\end{description}


\end{document}

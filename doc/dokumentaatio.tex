\documentclass[a4paper,titlepage]{article}
\usepackage{fancyhdr}
\usepackage{hyperref}
\usepackage[utf8]{inputenc}
\usepackage[T1]{fontenc}
\usepackage[finnish]{babel}

\hyphenation{kahvi-päivä-kirjaan}

\title{Kahvipäiväkirja}
\author{Miikka Koskinen \\ \url{miikka.koskinen@iki.fi}}

\begin{document}
\maketitle
\tableofcontents
\newpage

\section{Johdanto}

Juon mielelläni hyvää kahvia. En kuitenkaan oikeastaan osaa sanoa, että mikä
tekee kahvista hyvää tai millaisesta kahvista pidän. Siksi aion alkaa pitämään
kahvipäiväkirjaa, johon merkitsen, mitä kahvia olen juonut ja miltä se maistui.
Tämän työn tavoitteena on toteuttaa web-pohjainen kahvipäiväkirja.

Ohjelmointikielenä on Clojure. Clojure on moderni, dynaaminen,
Lisp-tyylinen kieli. Clojure toimii JVM-alusta, joten Clojurella
tehdyt web-sovelluksia on helppo ajaa esimerkiksi Tomcatilla.

Tietokantana toimii PostgreSQL.

\section{Yleiskuvaus}

Käyttäjäryhmät:

\begin{description}
    \item[Jokamies]
        Jokamies on kuka tahansa, joka vierailee kahvipäiväkirjassa. Kaikki
        muut käyttäjäryhmät kuuluvat myös tähän käyttäjäryhmään.

    \item[Maistelija]
        Maistelija on rekisteröity käyttäjä. Maistelija voi lisätä
        kahvipäiväkirjaan omia maistelukokemuksiaan.

    \item[Ylläpitäjä]
        Ylläpitäjä on rekisteröity käyttäjä, joka hallinnoi käyttäjätilejä ja
        joka pitää huolta siitä, että järjestelmään syötetty tieto on
        laadukasta.
\end{description}


\end{document}
